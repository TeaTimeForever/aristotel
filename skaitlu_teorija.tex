\section{Skaitlu teorija}
Smotrovs Jurijs \\

\subsubsection{Abstraktā algebra}
\warn{Algebraiska struktura} -- kopa ar tajā definētām darbībam (piem. $\langle K, +, * \rangle$).
Darbības apraksts -- ir \warn{aksiomas}. Tas ko var izvest no aksiomam -- ir \warn{teorēmas}. \\

$\langle K, \circ \rangle$
\begin{itemize}
  \item G1)  $\forall x,y \in K$ $\exists! / z \in K (x \circ y = z)$
  \item G2) $\forall x, y, z \in K (x \circ (y \circ z) = (x \circ y) \circ z)$ -- \warn{associācija}
  \item Ja izpildas G1 un G2 -- tadu kopu sauc par \term{pusgrupu}
  \item G3) $\exists n \in K \forall x \in K (n \circ x = x \circ n = x)$ -- \warn{neitrālais elements}
  \item ja izpildas G1, G2, G3 -- to sauc par \term{monoīdu}
  \item G4) $\forall x \in K$  $\exists d_x \in K (x \circ d_x = d_x \circ x = n)$
  \item Ja izpildas G1, G2, G3 un G4 -- to sauc par \term{grupu}
  \item G5) $\forall x, y \in K $ $(x \circ y = y \circ x)$ \warn{komutatīvitāte}
  \item Ja izpildas G1 - G5 -- to sauc par \term{Ābela grupu}, vai \term{komutatīvu} grupu.
\end{itemize}

\subsubsection{Gredzens}
Gredzens -- ir struktura $\langle G, +, * \rangle$ kur izpildas sekojošas īpašības:
\begin{enumerate}
  \item $\langle G, + \rangle$ ir Ābela grupa
  \item $\langle G, * \rangle$ ir pusgruppa
  \item $\forall x,y, z \in G $ $(x * (y + z) = x * y + x * z)$
  \item $\forall x, y, z \in G$ $( (y + z) * x = y * x + z * x)$
\end{enumerate}
$$\underbrace{1 + 1 + 1 + ... + 1 = 0}_\text{m}$$
\textbf{$m$}-- gredzena raksturojums (harakteristika) \\
\textbf{nulles dalītāji}: tādi a un b, ka $a \ne 0, b \ne 0 $ un $a * b = 0$ \\
\textbf{apgriežams elements}: tāds $a$, kuram $\exists d_a: a * d_a = d_a * a = 1$ \\
\textbf{apgriežamo elementu kopa}: $U(G)$\\
Teorēma: $\langle U(G), * \rangle$ ir grupa.

\subsection{Lauks}
$\langle G, +, * \rangle$  -- komutatīvais gredzens ar vieninieku. \\
$G^* = G /\ \{0\}$\\
G4' -- $\forall x \in G^*$\\
Piemeri
\begin{enumerate}
  \item $\langle N, +, *\rangle$ - nav
  \item $\langle Z, +, *\rangle$ - kom. gredzens ar 1
  \item $\langle Q, +, *\rangle$ - lauks
  \item $\langle R, +, *\rangle$ - lauks
  \item $\langle C, +, *\rangle$ - lauks
\end{enumerate}

\term{Def} Par skaitla $\boldsymbol{a}$ \warn{atlikumu} klasi pēc modula
 $\boldsymbol{m}$ sauc visu to veselo skaitlu kopu, kuri ,
 dalot ar $\boldsymbol{m}$, dod atlikumu $\boldsymbol{a}$.

  Piemers. Pēc modula \term{7}:
  \begin{itemize}
    \item $\bar{0} = \{..., -14, -7, 0, 14, 21, 28, 35\}$
    \item $\bar{1} = \{..., -20, -13, -6, 1, 8, 15 ...\}$
    \item $\bar{18} = \bar{4} = \bar{11} = \{..., -10, -3, 4, 11, 18, 25\}$
    \item $\bar{2} + \bar{3} = \bar{5}$ -- panemot kopu kur visi atlikumi 
    ir 2 un saskaitit ar kopu kur visi atlikumi ir 3, tad dabusim kopu,
    kur visi atlikumi ir 5.
    
    \item $\bar{2} * \bar{3} = \bar{6}$ -- panemot kopu kur visi atlikumi
    ir 2 un reizinat to ar kopu kur visi atlikumi ir 3, tad dabusim kopu,
    kur visi atlikumi ir 6.
    
    \item $\bar{4} + \bar{6} = \bar{3}$  
    ($4 + 6 = 10; 10 / 7 \rightarrow modulis = 3$)
    
    \item $\bar{4} * \bar{6} = \bar{3}$ 
    ($4 + 6 = 24; 24 / 7 \rightarrow modulis = 3$)
  \end{itemize}

  \term{Def} Saka ka veseli skaitli $a$ un $b$ ir kongruenti, pec modula 
  $\boldsymbol{m}$, ja skaitlis $a-b$ dalās ar $\boldsymbol{m}$. To Pieraksta 
  $ a \equiv b (mod \ m)$

  $18 \equiv 4 (mod \ 7)$
\newline
  $Z_m$ -- atlikumi pec modula $\boldsymbol{m}$ \\
  $Z_m = \{ \bar{0}, \bar{1}, \bar{2} ...\bar{(m-1)}\}$ \\


  $\langle Z_m, +, * \rangle$ -- komutatīvais gredzens ar 1\\
  neitralais elements (+)  $\bar{0} \rightarrow\bar{0} + x = x$ \\
  neitralais elements (*)  $\bar{1} \rightarrow\bar{1} * x = x$ \\

Var pieradit ka nav 0 dalitāju.
$$a * b = 0$$
$$a^{-1}* a * b = a^{-1}*0$$
$$ 1*b = 0$$
$$ b = 0  \  \rightarrow pretruna$$ -- 


\begin{itemize}
  \item $mod \ 7: \{ \bar{0}, \bar{1}, \bar{3}, \bar{4}, \bar{5}, \bar{6}\}$
  \item $mod \ 6: \{ \bar{0}, \bar{1}, \bar{3}, \bar{4}, \bar{5} \}$ \\
  $\bar{2} * \bar{3} =_6 \bar{0} $ 
\end{itemize}
$Z_p = \{ \bar{0}, \bar{1}, \bar{2}, ..., \bar{p-1} \} - 2, 3, 5, 7, 11, 13 (primes) $ \\
$\forall a \ne 0 \ \exists \ d(\bar{a} * \bar{d} = \bar{1})$ \\
$\bar{0}, \bar{1}, \bar{2}, ..., \bar{p-1}$ \\
$\bar{0} * \bar{a}, \bar{1} * \bar{a}, \bar{2} * \bar{a}, ..., \bar{p-1} * \bar{a}$ \\
$0*a, 1*a, 2*a, ..., (p-1)*a$

Pienemsim pretejo: $k*a \equiv l * a (mod \ p), k \ne l , 0 \le k, l \le p-1 )$ \\
$(k -l)*a \equiv 0 (mod \ p)$ \\
$(k - l) * a$ dalas ar $p$ \\
$k -l $  dalas ar $p$ vai $a$ dalas ar p 

//  congruences, atlikumu kopu ipasibas





