\section{Skaitlu teorija}
Smotrovs Jurijs \\

\subsubsection{Abstraktā algebra}
\warn{Algebraiska struktura} -- kopa ar tajā definētām darbībam (piem. $\langle K, +, * \rangle$).
Darbības apraksts -- ir \warn{aksiomas}. Tas ko var izvest no aksiomam -- ir \warn{teorēmas}. \\

$\langle K, \circ \rangle$
\begin{itemize}
  \item G1)  $\forall x,y \in K$ $\exists! / z \in K (x \circ y = z)$
  \item G2) $\forall x, y, z \in K (x \circ (y \circ z) = (x \circ y) \circ z)$ -- \warn{associācija}
  \item Ja izpildas G1 un G2 -- tadu kopu sauc par \term{pusgrupu}
  \item G3) $\exists n \in K \forall x \in K (n \circ x = x \circ n = x)$ -- \warn{neitrālais elements}
  \item ja izpildas G1, G2, G3 -- to sauc par \term{monoīdu}
  \item G4) $\forall x \in K$  $\exists d_x \in K (x \circ d_x = d_x \circ x = n)$
  \item Ja izpildas G1, G2, G3 un G4 -- to sauc par \term{grupu}
  \item G5) $\forall x, y \in K $ $(x \circ y = y \circ x)$ \warn{komutatīvitāte}
  \item Ja izpildas G1 - G5 -- to sauc par \term{Ābela grupu}, vai \term{komutatīvu} grupu.
\end{itemize}

\subsubsection{Gredzens}
Gredzens -- ir struktura $\langle G, +, * \rangle$ kur izpildas sekojošas īpašības:
\begin{enumerate}
  \item $\langle G, + \rangle$ ir Ābela grupa
  \item $\langle G, * \rangle$ ir pusgruppa
  \item $\forall x,y, z \in G $ $(x * (y + z) = x * y + x * z)$
  \item $\forall x, y, z \in G$ $( (y + z) * x = y * x + z * x)$
\end{enumerate}
$$\underbrace{1 + 1 + 1 + ... + 1 = 0}_\text{m}$$
\textbf{$m$}-- gredzena raksturojums (harakteristika) \\
\textbf{nulles dalītāji}: tādi a un b, ka $a \ne 0, b \ne 0 $ un $a * b = 0$ \\
\textbf{apgriežams elements}: tāds $a$, kuram $\exists d_a: a * d_a = d_a * a = 1$ \\
\textbf{apgriežamo elementu kopa}: $U(G)$\\
Teorēma: $\langle U(G), * \rangle$ ir grupa.



