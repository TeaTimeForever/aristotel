\section{Datoru tikli I}
Guntars Bārzdińš \\
guntis@latnet.lv \\
331 kab.

\begin{center}
  \begin{tabular}{ l | l | l }
    Starpprocexxoru attalums & Location type  & Network type\\
  \hline
  0.1m   & Board \\
  1m     & System   & LAN\\
  10m    & Room     & LAN\\
  100m   & Building & LAN (Local Area Network)\\
  1000m  & Campus   & MAN\\
  10km   & City     & MAN (Metropolitan Area Netpwork)\\
  100km  & Country  & WAN (Wide Area Netpwork)\\
  1000km & Continent& WAN \\
  10000km& Planet   & WAN \\
  \end{tabular}
\end{center}

\term{LAN} -- Local Area Network.
Atrumi: (1Gbps, 10Gbps, 100Gbps).
Attalums: vitais paris - 200m, optiskais -- 40-70km.
\\

\term{MAN} -- Metropolitan Area Netpwork.
Galvena atškirība no LAN: ātrums, \underline{kam pieder}, izmaksas.


Datoru tikls sastāv no: dators un marsrutezators (router) + sakaru kanāli.
Sakaru kanālu tipi:
\begin{enumerate}
  \item point-to-point
  \item broadcast
\end{enumerate}

Tīklu topologijas. (Klasiskas ir zvaignze un rinkins)

ISO -- International Standarts Organisation

OSI - Open System Interconnection

\begin{center}
  \begin{tabular}{ r | l }
  \hline
  Application & Lietojuma līmenis\\
  Presentation& Datu reprezentācijas līmenis (kura formata jpeg/mp3/txt etc\\
  Session     & Sessijas līmenis   \\
  Transport   & Transporta līmenis -- nočeko ka visas paketas 
  kuram vajadzeja nosutities atnaca. Un parbauda ka fails bija tieks parsutits bez kludam.\\
  Network     & Tīkla līmenis (point-to-point) uzradas adresācija.
  Kam un ko atsutit. Vienīgais liīmenis kur nav iluzijas par to 
  ka connection ir peer-to-peer. Tas ir layers kur tiesi sazinas
   2 datori. \\
  DataLink    & Kanala līmenis (griež datus paketos), ar check summam\\
  Physical    & Fiziskais līmenis (biti) \\
  \end{tabular}
\end{center}

\term{SAP} -- Service Access Point, sanem datu paketes un suta talak vai pieprasa datus velreiz.
Nodrosina sakaru starp layeriem.

