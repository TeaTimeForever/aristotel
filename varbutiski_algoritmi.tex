\section{Varbutiskie algoritmi}
Andris Ambainis \\
Raina Bulvaris 19, 319 kab.\\
konsultācija piektdiena 14:30 -- 16:30 \\
\newline
grāmatas:
\begin{enumerate}
  \item M.Mitzenmacher, E.Upfal - Probability and Computing
  \item piezīmes e-stūdijās
\end{enumerate}

Atzīme: 40\% eksāmens + 60\% mājas darbi

\subsection{Piemērs varbutiskajam algoritmam}
\subsubsection{Polinomu vienādiibas pārbaude}
Dots: 2 polinomi $f(x)$ un $g(x)$ ar pakāpi $\le n$
$$ f(x) = (x^2 + 3)(x-4)+7 $$
$$ g(x) = (x+2) (x-3) (x+4) $$

\begin{enumerate}
  \item Pirmais veids: atrisināt polinomus vienkaršakus formus
  $$ f(x) = x^3 - 4x^2 + 3x -5 $$
  $$ g(x) = x^3 + 3x^2 - 10x -24 $$

  \item Varbutiskai algoritms
  \begin{itemize}
    \item izvelas $x \in \{1, ... 10n\}$
    \item izrekina $f(x), g(x)$
    \item ja $f(x) \ne g(x) \rightarrow $ seicina ka $f \ne g$
    \item ja $f(x) = g(x) \rightarrow $ secina ka $ f = g$
  \end{itemize}
  \warn{Algoritms var izdot atbildi ka $f = g$ arī tad, ja $f(x) \ne g(x)$.
  Jautājums -- cik bieži tas notiek?}
  
\end{enumerate}

