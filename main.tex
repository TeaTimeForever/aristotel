\documentclass{article}
  \usepackage[utf8]{inputenc}
  \usepackage{xcolor}
  \usepackage{amsmath}
  \usepackage{graphicx}
  \usepackage{subcaption}
  \usepackage{float}
  \usepackage{xargs}
  
  \newcommand{\term}[1]{\colorbox{pink}{#1}}
  \newcommand{\warn}[1]{\colorbox{yellow}{#1}}
  \newcommandx*\pic[4][4=default]{
    \begin{figure}[H]
      \centering
      \includegraphics[{#1}]{{#2}}
      \caption{{#3}}
      \label{#4}
    \end{figure}
    }
\newcommandx*{\picsPreview}[6][6=default]{
\begin{figure}[H]
	\centering
    \begin{subfigure}[b]{0.5\textwidth}
            \includegraphics[width=\textwidth]{#1}
            \caption{#2}
    \end{subfigure}%
    \begin{subfigure}[b]{0.5\textwidth}
            \includegraphics[width=\textwidth]{#3}
            \caption{#4}
    \end{subfigure}
    \caption{#5}
    \label{#6}
\end{figure}
}
  
  \title{Latvijas Universitāte}
  \author{Karina Pilusonoka}
  \date{September 2018}
  
  \begin{document}
  
  \maketitle
  \tableofcontents

  \section{Datoru tikli I}
Guntars Bārzdińš \\
guntis@latnet.lv \\
331 kab.

\begin{center}
  \begin{tabular}{ l | l | l }
    Starpprocexxoru attalums & Location type  & Network type\\
  \hline
  0.1m   & Board \\
  1m     & System   & LAN\\
  10m    & Room     & LAN\\
  100m   & Building & LAN (Local Area Network)\\
  1000m  & Campus   & MAN\\
  10km   & City     & MAN (Metropolitan Area Netpwork)\\
  100km  & Country  & WAN (Wide Area Netpwork)\\
  1000km & Continent& WAN \\
  10000km& Planet   & WAN \\
  \end{tabular}
\end{center}

\term{LAN} -- Local Area Network.
Atrumi: (1Gbps, 10Gbps, 100Gbps).
Attalums: vitais paris - 200m, optiskais -- 40-70km.
\\

\term{MAN} -- Metropolitan Area Netpwork.
Galvena atškirība no LAN: ātrums, \underline{kam pieder}, izmaksas.


Datoru tikls sastāv no: dators un marsrutezators (router) + sakaru kanāli.
Sakaru kanālu tipi:
\begin{enumerate}
  \item point-to-point
  \item broadcast
\end{enumerate}

Tīklu topologijas. (Klasiskas ir zvaignze un rinkins)

ISO -- International Standarts Organisation

OSI - Open System Interconnection

\begin{center}
  \begin{tabular}{ r | l }
  \hline
  Application & Lietojuma līmenis\\
  Presentation& Datu reprezentācijas līmenis (kura formata jpeg/mp3/txt etc\\
  Session     & Sessijas līmenis   \\
  Transport   & Transporta līmenis -- nočeko ka visas paketas 
  kuram vajadzeja nosutities atnaca. Un parbauda ka fails bija tieks parsutits bez kludam.\\
  Network     & Tīkla līmenis (point-to-point) uzradas adresācija.
  Kam un ko atsutit. Vienīgais liīmenis kur nav iluzijas par to 
  ka connection ir peer-to-peer. Tas ir layers kur tiesi sazinas
   2 datori. \\
  DataLink    & Kanala līmenis (griež datus paketos), ar check summam\\
  Physical    & Fiziskais līmenis (biti) \\
  \end{tabular}
\end{center}

\term{SAP} -- Service Access Point, sanem datu paketes un suta talak vai pieprasa datus velreiz.
Nodrosina sakaru starp layeriem.

\subsection{Majas darbs I}
\subsubsection{Fourier Analysis}
name: kaRina \\
R in ascii 01010010\\

$$f(t) = \frac{c}{2} + 
    \sum_{n=1}^{\infty}a_n cos(\frac{2 \pi n t}{T}) + 
    \sum_{n=1}^{\infty}b_n sin(\frac{2 \pi n t}{T})$$,
    where

$a_n=\frac{2}{T} \int_0^T f(t) cos (\frac{2 \pi n t}{T})dt $ and
$b_n=\frac{2}{T} \int_0^T f(t) sin (\frac{2 \pi n t}{T})dt $
$$c = \frac{3}{4}$$

$$a_n = \frac{1}{\pi n } \bigg[ 
  sin(\frac{2 \pi n}{4}) -
  sin(\frac{\pi n}{4})   +
  sin(\frac{4 \pi n}{4}) -
  sin(\frac{3 \pi n}{4}) +
  sin(\frac{7 \pi n}{4}) -
  sin(\frac{6 \pi n}{4})
 \bigg]
 $$

 $$b_n = \frac{1}{\pi n } \bigg[ 
  cos(\frac{\pi n}{4}) -
  cos(\frac{2 \pi n}{4})   +
  cos(\frac{3 \pi n}{4}) -
  cos(\frac{4 \pi n}{4}) +
  cos(\frac{6 \pi n}{4}) -
  cos(\frac{7 \pi n}{4})
 \bigg]
 $$

 \begin{center}
  \begin{tabular}{ c | c | c  }
    harmonika ($n$) & $a_n$ & $b_n$\\
  \hline
  1 & 
    $\frac{4 - 3 \sqrt{2}}{2 \pi}$ &
    $\frac{2 - \sqrt{2}}{2 \pi} $ \\
  2 & 
    $-\frac{1}{2\pi}$ &
    $-\frac{1}{2\pi}$ \\
  3 & 
    $\frac{-4 - 3 \sqrt{2}}{6 \pi}$ &
    $\frac{2 + \sqrt{2}}{6 \pi}$ \\
  4 &
    0 &
    $-\frac{1}{2 \pi}$ \\
  5 &
    $\frac{4 + 3 \sqrt{2}}{10 \pi}$ &
    $\frac{2 + \sqrt{2}}{10 \pi}$ \\
  6 &
    $\frac{1}{6 \pi}$ &
    $-\frac{1}{6 \pi}$ \\
  7 &
    $\frac{3 \sqrt{2} - 4}{14 \pi}$ &
    $ \frac{2 - \sqrt{2}}{14 \pi} $ \\
  8 & 0 & 0\\
  9 &
    $\frac{4 - 3 \sqrt{2}}{18 \pi}$ &
    $\frac{2 - \sqrt{2}}{18 \pi} $ \\
  10&
    $ -\frac{1}{10 \pi}$ &
    $ -\frac{1}{10 \pi}$ \\
  \end{tabular}
\end{center}




  \section{Varbutiskie algoritmi}
Andris Ambainis \\
Raina Bulvaris 19, 319 kab.\\
konsultācija piektdiena 14:30 -- 16:30 \\
\newline
grāmatas:
\begin{enumerate}
  \item M.Mitzenmacher, E.Upfal - Probability and Computing
  \item piezīmes e-stūdijās
\end{enumerate}

Atzīme: 40\% eksāmens + 60\% mājas darbi

\subsection{Piemērs varbutiskajam algoritmam}
\subsubsection{Polinomu vienādiibas pārbaude}
Dots: 2 polinomi $f(x)$ un $g(x)$ ar pakāpi $\le n$
$$ f(x) = (x^2 + 3)(x-4)+7 $$
$$ g(x) = (x+2) (x-3) (x+4) $$

\begin{enumerate}
  \item Pirmais veids: atrisināt polinomus vienkaršakus formus
  $$ f(x) = x^3 - 4x^2 + 3x -5 $$
  $$ g(x) = x^3 + 3x^2 - 10x -24 $$

  \item Varbutiskai algoritms
  \begin{itemize}
    \item izvelas $x \in \{1, ... 10n\}$
    \item izrekina $f(x), g(x)$
    \item ja $f(x) \ne g(x) \rightarrow $ seicina ka $f \ne g$
    \item ja $f(x) = g(x) \rightarrow $ secina ka $ f = g$
  \end{itemize}
  \warn{Algoritms var izdot atbildi ka $f = g$ arī tad, ja $f(x) \ne g(x)$.
  Jautājums -- cik bieži tas notiek?}
  
\end{enumerate}


  \section{Atru algoritmu konstruēšana}
Viksna \\

Problemas piemēri: Eulera tilti, Hamiltona cycles

% https://www.geogebra.org/graphing
% \pic{width=10cm}{images/a.png}{track}
  \section{Data processing}
Girts Karnitis

  

  % \pic{width=7cm}{src}{Caption}
  % \picsPreview
  % {img/svd_vect/ab_mod.png}{трехмерная поверхность}
  % {img/svd_vect/ab_mod_marg.png}{линии уровня}
  % {$\tilde{F}(a,b) = |100 - (ab)|$ на интервале $a \in [-17, 17]$ и $b \in [-17,
  % 17]$.}
  \end{document}
  